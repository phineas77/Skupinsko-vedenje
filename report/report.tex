\documentclass[9pt]{pnas-new}
% Use the lineno option to display guideline numbers if required.
% Note that the use of elements such as single-column equations
% may affect the guideline number alignment. 

\RequirePackage[english]{babel} % when writing in english

\templatetype{pnasresearcharticle} % Choose template 
% {pnasresearcharticle} = Template for a two-column research article
% {pnasmathematics} = Template for a one-column mathematics article
% {pnasinvited} = Template for a PNAS invited submission

\newcommand{\set}[1]{\ensuremath{\mathbf{#1}}}
\renewcommand{\vec}[1]{\ensuremath{\mathbf{#1}}}
\newcommand{\uvec}[1]{\ensuremath{\hat{\vec{#1}}}}
\newcommand{\const}[1]{{\ensuremath{\kappa_\mathrm{#1}}}} 
\newcommand{\num}[1]{#1}

\graphicspath{{./fig/}}

\title{Reimplementing the FRIsheeping herding algorithm using fuzzy logic}

% Use letters for affiliations, numbers to show equal authorship (if applicable), and to indicate the corresponding author
\author{Noah Novšak}
\author{Veljko Dudić}
\author{Petra Kuralt}
\author{Timotej Košir}

\affil{Collective behavior course research seminar report} 

% Please give the surname of the lead author for the running footer
% \leadauthor{}

\selectlanguage{english}

% Please add here a significance statement to explain the relevance of your work
\significancestatement{A herding algorithm that uses fuzzy logic}{Lorem ipsum.}{keywords}

% Please include the corresponding author, author contribution, and author declaration information
%\authorcontributions{Please provide details of author contributions here.}
%\authordeclaration{Please declare any conflict of interest here.}
%\equalauthors{\textsuperscript{1}A.O.(Author One) and A.T. (Author Two) contributed equally to this work (remove if not applicable).}
%\correspondingauthor{\textsuperscript{2}To whom correspondence should be addressed. E-mail: author.two\@email.com}

% Keywords are not mandatory, but authors are strongly encouraged to provide them. If provided, please include two to five keywords, separated by the pipe symbol, e.g.:
\keywords{keywords} 

\begin{abstract}
Lorem ipsum.
\end{abstract}

\dates{\textbf{\today}}
\program{BM-RI}
\vol{2018/19}
\no{CB:G1} % group ID
%\fraca{FRIteza/201516.130}

\begin{document}

% Optional adjustment to line up main text (after abstract) of the first page with line numbers when using both lineno and twocolumn options.
% You should only change this length when you've finalized the article contents.
\verticaladjustment{-2pt}

\maketitle
\thispagestyle{firststyle}
\ifthenelse{\boolean{shortarticle}}{\ifthenelse{\boolean{singlecolumn}}{\abscontentformatted}{\abscontent}}{}

% If your first paragraph (i.e., with the \dropcap) contains a list environment (quote, quotation, theorem, definition, enumerate, itemize...), the line after the list may have some extra indentation. If this is the case, add \parshape=0 to the end of the list environment.
\dropcap{I}ntroduction
\cite{shepherdingalgorithm}
\cite{herdingalgorithm}
\cite{fuzzymodel}
\cite{flocktransport}
\cite{crowdmodelling}

\section*{Methods}

\subsection*{Some method}

\subsection*{Some other method}

\subsection*{Implementation}

\section*{Results}

\section*{Discussion}

\acknow{distribution of labor}
\showacknow % Display the acknowledgments section

% \pnasbreak splits and balances the columns before the references.
% If you see unexpected formatting errors, try commenting out this line
% as it can run into problems with floats and footnotes on the final page.
%\pnasbreak

\begin{multicols}{2}
\section*{\bibname}
% Bibliography
\bibliography{./bib/bibliography}
\end{multicols}

\end{document}