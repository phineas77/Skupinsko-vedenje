\documentclass[9pt]{pnas-new}

\RequirePackage[english]{babel}

\templatetype{pnasresearcharticle}

\newcommand{\set}[1]{\ensuremath{\mathbf{#1}}}
\renewcommand{\vec}[1]{\ensuremath{\mathbf{#1}}}
\newcommand{\uvec}[1]{\ensuremath{\hat{\vec{#1}}}}
\newcommand{\const}[1]{{\ensuremath{\kappa_\mathrm{#1}}}} 
\newcommand{\num}[1]{#1}

\graphicspath{{./fig/}}

\title{Reimplementing the FRIsheeping herding algorithm using fuzzy logic}

\author{Veljko Dudić}
\author{Noah Novšak}
\author{Petra Kuralt}
\author{Timotej Košir}

\affil{Collective behavior course research seminar report} 

\selectlanguage{english}

\significancestatement{A herding algorithm that uses fuzzy logic}{Summary of Findings: Recapitulate key findings and their significance.}{fuzzy logic | herding algorithm}

\equalauthors{\textsuperscript{1}N.N.(Noah Novšak), V.D. (Veljko Dudić), P.K. (Petra Kuralt), and T.K. (Timotej Košir) contributed equally to this work.}

\keywords{fuzzy logic | herding algorithm} 

\begin{abstract}
Concisely summarize the research objectives, methodology, key findings, and implications.
\end{abstract}

\dates{\textbf{\today}}
\program{BM-RI}
\vol{2023/24}
\no{CB:G} % group ID

\begin{document}

% You should only change this length when you've finalized the article contents.
\verticaladjustment{-2pt}

\maketitle
\thispagestyle{firststyle}
\ifthenelse{\boolean{shortarticle}}{\ifthenelse{\boolean{singlecolumn}}{\abscontentformatted}{\abscontent}}{}

% If your first paragraph (i.e., with the \dropcap) contains a list environment (quote, quotation, theorem, definition, enumerate, itemize...), the line after the list may have some extra indentation. If this is the case, add \parshape=0 to the end of the list environment.
\dropcap{I}ntroduction: Briefly introduce the background, i.e., the concept of herding algorithms and their applications.

Motivation: Discuss the need for reimplementing herding algorithms with fuzzy logic.

Research Question: Formulate the main research question and objectives.

\subsection*{Literature review}
Overview of Herding Algorithms: Summarize existing algorithms, their strengths, and limitations. \cite{shepherdingalgorithm} \cite{optimalshepherding}

Fuzzy Logic in Algorithms: Explore previous studies incorporating fuzzy logic into algorithms. \cite{fuzzylogic} \cite{modellingcrowd}

\section*{Methodology}
Selection of Herding Algorithm: Justify the choice of the specific herding algorithm for implementation. \cite{behaviorbased}

Integration of Fuzzy Logic: Describe the process of incorporating fuzzy logic into the chosen algorithm.

Data Collection: Detail the environment used for testing and evaluating the reimplemented algorithm.

Performance Metrics: Specify the metrics used to assess the algorithm's performance.

\section*{Results}
Implementation Details: Present the technical aspects of the reimplemented herding algorithm with fuzzy logic.

Comparative Analysis: Compare the new algorithm's performance with the original version and other relevant benchmarks.

Discussion of Results: Interpret the findings and discuss any unexpected outcomes.

\section*{Discussion}
Implications: Explore the potential applications and benefits of the reimplemented algorithm.

Limitations: Discuss any limitations of the study and potential areas for future research.

Practical Considerations: Address practical aspects of implementing the algorithm in real-world scenarios.

\acknow{distribution of labor}
\showacknow % Display the acknowledgments section

% If you see unexpected formatting errors, try commenting out this line
% as it can run into problems with floats and footnotes on the final page.
% \pnasbreak % Splits and balances the columns before the references.

\begin{multicols}{2}
\section*{\bibname}
% Bibliography
\bibliography{./bib/bibliography}
\end{multicols}

\end{document}